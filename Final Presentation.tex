\documentclass[aspectratio=169]{beamer}
%\usecolortheme{beaver}
\usepackage[english,ukrainian]{babel}
\usepackage{hyperref}
\hypersetup{colorlinks,urlcolor=blue}
\makeatletter
\DeclareUrlCommand\ULurl@@{%
  \def\UrlFont{\ttfamily\color{blue}}%
  \def\UrlLeft{\uline\bgroup}%
  \def\UrlRight{\egroup}}
\def\ULurl@#1{\hyper@linkurl{\ULurl@@{#1}}{#1}}
\DeclareRobustCommand*\ULurl{\hyper@normalise\ULurl@}
\makeatother
\usepackage[table]{colortbl}
\usecolortheme{whale}
\usepackage{etoolbox,refcount}
\usepackage{multicol}

\newcounter{countitems}
\newcounter{nextitemizecount}
\newcommand{\setupcountitems}{%
  \stepcounter{nextitemizecount}%
  \setcounter{countitems}{0}%
  \preto\item{\stepcounter{countitems}}%
}
\makeatletter
\newcommand{\computecountitems}{%
  \edef\@currentlabel{\number\c@countitems}%
  \label{countitems@\number\numexpr\value{nextitemizecount}-1\relax}%
}
\newcommand{\nextitemizecount}{%
  \getrefnumber{countitems@\number\c@nextitemizecount}%
}
\newcommand{\previtemizecount}{%
  \getrefnumber{countitems@\number\numexpr\value{nextitemizecount}-1\relax}%
}
\makeatother    
\newenvironment{AutoMultiColItemize}{%
\ifnumcomp{\nextitemizecount}{>}{3}{\begin{multicols}{2}}{}%
\setupcountitems\begin{itemize}}%
{\end{itemize}%
\unskip\computecountitems\ifnumcomp{\previtemizecount}{>}{3}{\end{multicols}}{}}
\usepackage[utf8]{inputenc}

\title{Використання методів deep learning для прогнозування динаміки фондових індексів}
\author{Тараба Віктор\\ Науковий керівник: д.е.н., доц. Ставицький А. В.}
\institute{Київський національний університет імені Тараса Шевченка\\Економічний факультет\\Кафедра економічної кібернетики}

\begin{document}
	
\begin{frame}
\titlepage
\end{frame}

\begin{frame}
\frametitle{План презентації}
\setbeamertemplate{section in toc}[circle]
%\setbeamercolor{section in toc}{fg=UniBlue}
%\setbeamercolor*{section in toc}{fg=UniBlue}
\tableofcontents
\end{frame}

\section{Опис дослідження та вхідних даних}

\begin{frame}
\frametitle{Опис дослідження}
\begin{itemize}
\item \alert {Метою дослідження} є аналіз динаміки фондових індексів та аналіз прибутковості торгових стратегій, що базуються на прогнозах моделей штучних нейронних мереж.
\bigskip
\item \alert {Об’єктом дослідження} є фондові індекси 4 країн (США – S\&P 500, КНР – SSE Composite Index, Південна Корея – Korea Composite Stock Price Index, Японія – Nikkei225).
\bigskip
\item \alert {Предметом дослідження} є методи прогнозування фондових індексів та їх практичне застосування.

\end{itemize}
\end{frame}

\begin{frame}
\frametitle{Вхідні дані}
\begin{itemize}
\item \alert {Дані} (Open, High, Low, Close, Volume) з 2000 по 2021 рр. зі щоденною періодичністю;  завантажено за допомогою біблотеки yfinance (з Yahoo! Finance) \\
\bigskip
\item На основі цих даних було розраховано такі показники:\\
\begin{itemize}
    \begin{AutoMultiColItemize}
    \item[\textcolor{orange}{\textbullet}] Percent change 1
    \item[\textcolor{orange}{\textbullet}] Percent change 5
    \item[\textcolor{orange}{\textbullet}] Percent change 30 
    \item[\textcolor{orange}{\textbullet}] Percent change Open 
    \item[\textcolor{orange}{\textbullet}] Percent change High 
    \item[\textcolor{orange}{\textbullet}] Percent change Low 
    \item[\textcolor{orange}{\textbullet}] Percent change Volume 
    \item[\textcolor{orange}{\textbullet}] DI
    \item[\textcolor{orange}{\textbullet}] ERI 
    \item[\textcolor{orange}{\textbullet}] SMA 1
    \item[\textcolor{orange}{\textbullet}] EMA 1 
    \item[\textcolor{orange}{\textbullet}] LWMA 1
    \item[\textcolor{orange}{\textbullet}] MAE 1
    \item[\textcolor{orange}{\textbullet}] MAE 2
    \item[\textcolor{orange}{\textbullet}] MAE 3 
    \item[\textcolor{orange}{\textbullet}] MAE 4
    \item[\textcolor{orange}{\textbullet}] MAE 5
    \item[\textcolor{orange}{\textbullet}] MAE 6
    \item[\textcolor{orange}{\textbullet}] CCI 1 
    \item[\textcolor{orange}{\textbullet}] SO 1
    \item[\textcolor{orange}{\textbullet}] CMO 1 
    \end{AutoMultiColItemize}
\end{itemize}
\end{itemize}
\end{frame}

\section{Навчання моделей}

\begin{frame}
\frametitle{Оптимізаційний алгоритм}
\[ v_{t+1} = \beta_{1} v_{t} + (1-\beta_{1})\nabla J(W_{t}) \]\\
\[ cache_{t+1} = \beta_{2} cache_{t} + (1-\beta_{2})(\nabla J(W_{t}))^{2} \]\\
\[ W_{t+1} = W_{t}-\frac{\eta}{\sqrt{cache_{t+1}+\varepsilon}}v_{t+1} \]
\bigskip
\begin{itemize}
\item \alert {Оптимізаційний алгоритм Adam} є стандартним вибором для більшості нейронних мереж. Значення параметрів – за замовчуванням ($\eta$=0.001, $\beta_{1}$=0.9, $\beta_{1}$=0.999, $\varepsilon$=1e-07). 
\end{itemize}
\end{frame}

\begin{frame}
\frametitle{Навчання моделі}
\begin{itemize}
\item \small Для роботи з нейронними мережами використано \alert {бібліотеки} Python \textit{tensorflow (keras)} та \textit{sklearn}.\\
\bigskip
\item \small \alert {Гіперпараметри}, які використовувалися при підборі моделей:\\
\begin{itemize}
\item[\textcolor{orange}{\textbullet}] \textit{nodes\_first} – к-ть нейронів в першому шарі; \textit{nodes\_first} $\in$ [4, 8, 16, 32];\\
\item[\textcolor{orange}{\textbullet}] \textit{nodes\_second} – к-ть нейронів в другому шарі; \textit{nodes\_second} $\in$ [4, 8, 16, 32];\\
\item[\textcolor{orange}{\textbullet}] \textit{batch} – к-ть елементів з набору даних, на яких «навчається» модель кожної епохи; \textit{batch} $\in$ [64, 128, 256, 512, 1024]\\ 
\item[\textcolor{orange}{\textbullet}] \textit{dropout\_include} – набуває значення 1, якщо при побудові моделі використовується dropout для регуляризації; 0 інакше;\\
\item[\textcolor{orange}{\textbullet}] \textit{rate} – параметр для dropout, відповідає за ймовірність виключення нейронів з мережі; \textit{rate} $\in$ [0.2, 0.3, 0.4, 0.5].\\
\bigskip
\end{itemize}
\item \small Точка, в якій варто зупинити навчання моделі, аби уникнути перенавчання, визначалася автоматично за допомогою функції \textit{tf.keras.callbacks.EarlyStopping}. Для вибору оптимальних моделей використовувалася метрика \textit{binary accuracy}. 
\end{itemize}
\end{frame}

\section{Результати торгових стратегій, які базуються на прогнозах нейронних мереж}

\begin{frame}
\frametitle{Результати торгових стратегій, які базуються на прогнозах НМ}
\begin{center}
Прибутковість стратегії #1 (індекс KOSPI) \\
\bigskip
\includegraphics[scale=0.3]{index.jpg}
\end{center}
\end{frame}

\begin{frame}
\frametitle{Порівняння результатів торгових стратегій}
Результати без врахування транзакційних витрат:\\
\bigskip
\begin{tabular}{ |p{7.8cm}||p{1.6cm}|p{1.6cm}|p{1.6cm}|  }
 \hline
 \multicolumn{4}{|c|}{S\&P 500} \\
 \hline
 Показник& Модель #1 &Модель #2&Модель #3\\
 \hline
 Загальний прибуток (\%)	&\cellcolor{pink!25}70.59&	\cellcolor{pink!25}80.02	&\cellcolor{pink!25}64.82\\
 Buy-and-hold (\%) & \cellcolor{blue!8}95.99 & \cellcolor{blue!8}95.99 & \cellcolor{blue!8}95.99 \\
 %\cellcolor{green!25}
 Максимальне значення капіталу &	175.26&	180.02	&167.14\\
Мінімальне значення капіталу&	93.01	&89.90	&90.91\\
Закрито позицій&	202&	130&	142\\
Максимальний прибуток за одну операцію (\%)&	12.48&	10.37&	12.48\\
Мінімальний прибуток за одну операцію (\%)&	-15.96&	-22.25&	-17.84\\
 \hline
\end{tabular}
\end{frame}

\begin{frame}
\frametitle{Порівняння результатів торгових стратегій}
Результати без врахування транзакційних витрат:\\
\bigskip
\begin{tabular}{ |p{7.8cm}||p{1.6cm}|p{1.6cm}|p{1.6cm}|  }
 \hline
 \multicolumn{4}{|c|}{KOSPI} \\
 \hline
 Показник& Модель #1 &Модель #2&Модель #3\\
 \hline
 Загальний прибуток (\%)&\cellcolor{pink!25}20.40&	\cellcolor{pink!25}4.89&	\cellcolor{pink!25}6.15\\
 Buy-and-hold (\%) & \cellcolor{blue!8}52.64 &\cellcolor{blue!8}52.64& \cellcolor{blue!8}52.64\\
 %\cellcolor{green!25}
 Максимальне значення капіталу &123.71&	104.89&	106.15	\\
Мінімальне значення капіталу&	72.66&	66.90&	54.77\\
Закрито позицій&	320&	369	&352\\
Максимальний прибуток за одну операцію (\%)&19.89&	14.08&	19.89	\\
Мінімальний прибуток за одну операцію (\%)&	-12.13&	-12.13&	-24.16\\
 \hline
\end{tabular}
\end{frame}

\begin{frame}
\frametitle{Порівняння результатів торгових стратегій}
Результати без врахування транзакційних витрат:\\
\bigskip
\begin{tabular}{ |p{7.8cm}||p{1.6cm}|p{1.6cm}|p{1.6cm}|  }
 \hline
 \multicolumn{4}{|c|}{Nikkei225} \\
 \hline
 Показник& Модель #1 &Модель #2&Модель #3\\
 \hline
 Загальний прибуток (\%)& \cellcolor{pink!25}29.09	&\cellcolor{pink!25}73.42&	\cellcolor{pink!25}40.31\\
 Buy-and-hold (\%) & \cellcolor{blue!8}85.05 &\cellcolor{blue!8}85.05& \cellcolor{blue!8}85.05\\
 %\cellcolor{green!25}
 Максимальне значення капіталу &	160.73&	175.50&	140.54\\
Мінімальне значення капіталу&	94.32&	95.56	&87.45\\
Закрито позицій&	477&	203	&251\\
Максимальний прибуток за одну операцію (\%)&6.98	&14.11&	14.39	\\
Мінімальний прибуток за одну операцію (\%)&	-19.96&	-13.53&	-13.53\\
 \hline
\end{tabular}
\end{frame}

\begin{frame}
\frametitle{Порівняння результатів торгових стратегій}
Результати без врахування транзакційних витрат:\\
\bigskip
\begin{tabular}{ |p{7.8cm}||p{1.6cm}|p{1.6cm}|p{1.6cm}|  }
 \hline
 \multicolumn{4}{|c|}{SSE} \\
 \hline
 Показник& Модель #1 &Модель #2&Модель #3\\
 \hline
 Загальний прибуток (\%)&\cellcolor{green!25}31.41&	\cellcolor{pink!25}-11.22&	\cellcolor{green!25}18.88\\
 Buy-and-hold (\%) & \cellcolor{blue!8}11.48 &\cellcolor{blue!8}11.48 &\cellcolor{blue!8}11.48\\
 %\cellcolor{green!25}
 Максимальне значення капіталу &	131.41&	116.06&	118.88\\
Мінімальне значення капіталу&	94.19&	84.86&	93.14\\
Закрито позицій&177&	161	&185	\\
Максимальний прибуток за одну операцію (\%)&	14.35&	21.53&	14.35\\
Мінімальний прибуток за одну операцію (\%)& -5.71&	-5.71&	-4.42	\\
 \hline
\end{tabular}
\end{frame}

\begin{frame}
\frametitle{Порівняння результатів торгових стратегій}
Результати з врахуванням транзакційних витрат*:\\
\bigskip
\begin{tabular}{ |p{8cm}||p{2.3cm}|p{2.3cm}| }
 \hline
 \multicolumn{3}{|c|}{SSE} \\
 \hline
 Показник& Модель #1 &Модель #3\\
 \hline
 Загальний прибуток (\%)&\cellcolor{green!25}20.72&	\cellcolor{pink!25}8.78\\
 Buy-and-hold (\%) & \cellcolor{blue!8}11.48 &\cellcolor{blue!8}11.48\\
 %\cellcolor{green!25}
 Максимальне значення капіталу &120.72	&108.78	\\
Мінімальне значення капіталу&90.14	&88.88	\\
Максимальний прибуток за одну операцію (\%)&	14.30	&14.30\\
Мінімальний прибуток за одну операцію (\%)&	-5.76	&-4.47\\
 \hline
\end{tabular}
\bigskip
\\
\textcolor{gray}{*Як оцінку значення транзакційних витрат взято середнє значення брокерської комісії для материкового Китаю відповідно до даних, наведених в звіті KPMG – 0.048\%.}
\end{frame}

\begin{frame}
\frametitle{Висновки}
\begin{itemize}
\item Навіть без врахування транзакційних витрат тільки 2 з 12 торгових стратегій дозволили «переграти» buy-and-hold. Врахування транзакційних витрат призвело до того, що результати лише однієї стратегії все ще залишалися кращими за buy-and-hold. 
\bigskip
\item Отже, використання найпростішої версії нейронних мереж \alert {\textbf{не дозволило б нам переграти ринок}} на тестових вибірках для 3 з 4 розглянутих в роботі фондових індексів.
\bigskip
\definecolor{cadmiumgreen}{rgb}{0.0, 0.42, 0.24}
\item \textcolor{blue} {У подальшому доцільно було б розглянути} інші методи машинного навчання, оновити оптимальні параметри для методів технічного аналізу., збільшити кількість індексів та додати до вхідних даних сигнали за більшою кількістю методів ТА 
\bigskip
\end{itemize}
\end{frame}

\section{Q\&A}

\begin{frame}
\frametitle{Q\&A}
\begin{center}
\textcolor{blue}{\huge Дякую за увагу!} \\
\end{center}
\bigskip
\bigskip
Дані та код (для обробки даних, навчання моделей, розрахунків та побудови графіків) доступні за посиланням: 
\ULurl{https://github.com/Victor-T2001/Term-Project-2021}
\end{frame}

\begin{frame}
\frametitle {Використання методів deep learning для прогнозування динаміки фондових індексів}
\setbeamertemplate{section in toc}[circle]
%\setbeamercolor{section in toc}{fg=UniBlue}
%\setbeamercolor*{section in toc}{fg=UniBlue}
\tableofcontents
\end{frame}

\end{document}
